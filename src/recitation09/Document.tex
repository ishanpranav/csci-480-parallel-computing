\documentclass[12pt]{article}
\usepackage[english]{babel}
\usepackage[letterpaper,top=2cm,bottom=2cm,left=3cm,right=3cm,marginparwidth=1.75cm]{geometry}
\usepackage{amsmath}
\usepackage{graphicx}
\title{CSCI-UA 201 Recitation 9}
\author{Ishan Pranav}
\date{March 31, 2023}
\begin{document}
\maketitle
\section{The average access time}
Let $p$ represent the probability of a cache hit, $m$ represent the cache access time, and $M$ represent the main memory access time.
\[p=95\%.\]
\[m=2.\]
\[M=150.\]

The average memory access time per reference is $A_{\text{MAT}}$.
\begin{align*}
A_{\text{MAT}}
&=mp+(M+m)(1-p)\\
&=(2)(95\%)+(150+2)(5\%)\\
&=9.5.
\end{align*}
The average memory access time for 100 references is $100A_{\text{MAT}}$, or 950 cycles.
\[100A_{\text{MAT}}=100\times 9.5=950.\]
\section{A medical application}
Let $p_t$ represent the probability of a memory access operation, $p$ represent the probability of a cache hit, $t$ represent the non-memory-access time, $m$ represent the cache access time, and $M$ represent the main memory access time.
\[p_t=30\%.\]
\[t=1.\]
\[p=80\%.\]
\[m=1.\]
\[M=20.\]

The average operations time is $A$, or 5.3 cycles per operation.
\begin{align*}
A
&=p_tt+mp+(M+m)(1-p)\\
&=(30\%)(1)+(1)(80\%)+(20+1)(1-80\%)\\
&=5.3.
\end{align*}
\section{The breakdown of the direct-mapped cache address}
The address size is 64 bits, the cache size is 32 kB, the block size is 64 bytes, and the cache is direct-mapped.

The offset size is 6 bits, required to address the 64 bytes per block.
\[\log_2(64)=6.\]

The number of blocks is 512.

\[\frac{32\text{ kB}}{64\text{ B}}\times\frac{1024\text{ B}}{1\text{ kB}}=\frac{32768\text{ B}}{64\text{ B}}=512.\]

The index size is 9 bits, required to address the 512 blocks.
\[\log_2(512)=9.\]

The address size is 64 bits. The least significant 6 bits represent the offset, the next 9 bits represent the index, and the remaining 49 bits represent the tag.
\section{The breakdown of the four-way set-associative cache address}
The address size is 24 bits, the cache size is 64 kB, the block size is 16 bytes, and the cache is four-way set-associative.

The number of blocks is 4096.
\[\frac{64\text{ kB}}{16\text{ B}}\times\frac{1024\text{ B}}{1\text{ kB}}=\frac{65536\text{ B}}{16\text{ B}}=4096.\]

The number of sets is 1024.
\[4096\text{ blocks}\times\frac{1\text{ set}}{4\text{ blocks}}=1024\text{ sets}.\]

The offset size is 4 bits, required to address the 16 bytes per block.
\[\log_2(16)=4.\]

The index size is 10 bits, required to address the 1024 sets.
\[\log_2(1024)=10.\]

The address size is 24 bits. The least significant 4 bits represent the offset, the next 10 bits represent the index, and the remaining 10 bits represent the tag.
\section{A small C program}
\begin{verbatim}
   #include <stdio.h>
\end{verbatim}

\verb|/**|

\verb| * |Computes the floor of the integer binary logarithm of the given number.

\verb| * |

\verb| * @param |n the number

\verb| * @return |The base-2 logarithm, rounded down to the nearest integer.

\verb| */|
\begin{verbatim}
   static int floorLog2(unsigned int n)
   {
      return 31 - __builtin_clz(n);
   }
\end{verbatim}

\verb|/**|

\verb| * |The main entry point for the application.

\verb| * |

\verb| * @return |An exit code.

\verb| */|
\begin{verbatim}
   int main()
   {
      int cacheSize, addressLength, blockSize, associativity;
    
      printf("Cache size\t(kB):\t");
      scanf("%d", &cacheSize);
      printf("Address length\t(bits):\t");
      scanf("%d", &addressLength);
      printf("Block size\t(B):\t");
      scanf("%d", &blockSize);
      printf("Associativity\t(-way):\t");
      scanf("%d", &associativity);
    
      int index = floorLog2((cacheSize * 1024 / blockSize) / associativity),
          offset = floorLog2(blockSize),
          tag = addressLength - (offset + index);
    
      printf("\nTag:\t%d\tbits\nIndex:\t%d\tbits\nOffset:\t%d\tbits\n",
         tag, index, offset);
         
      return 0;
   }
\end{verbatim}
\end{document}
